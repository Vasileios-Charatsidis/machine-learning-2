\documentclass{amsmlaj}
\begin{document}
\lecturesol{Homework 6}{Ke Tran}{m.k.tran@uva.nl}{21 May, 2016}
{Andrea Jemmett}{11162929}{andrea.jemmett@student.uva.nl}{N/A}


\paragraph{Problem 2.}
\begin{sol}
	\begin{itemize}
		\item[(a)] \hfill \vspace{-1cm}
			\begin{figure}[h]
				\centering
				\begin{subfigure}[b]{.3\textwidth}
					\centering
					\begin{tikzpicture}
						\node[obs] (x) at (0,0) {$X$};
						\node[obs] (y) at (1.5,0) {$Y$};
					\end{tikzpicture}
					\caption{No cause-effect relation}
					\label{fig:cbn1}
				\end{subfigure}
				\begin{subfigure}[b]{.3\textwidth}
					\centering
					\begin{tikzpicture}
						\node[obs] (x) at (0,0) {$X$};
						\node[obs] (y) at (1.5,0) {$Y$};
						\draw[->] (x) -- (y);
					\end{tikzpicture}
					\caption{$X$ causes $Y$}
					\label{fig:cbn2}
				\end{subfigure}
				\begin{subfigure}[b]{.3\textwidth}
					\centering
					\begin{tikzpicture}
						\node[obs] (x) at (0,0) {$X$};
						\node[obs] (y) at (1.5,0) {$Y$};
						\draw[->] (y) -- (x);
					\end{tikzpicture}
					\caption{$Y$ causes $X$}
					\label{fig:cbn3}
				\end{subfigure}
				\caption{Two nodes Causal Bayesian Networks}
				\label{fig:cbn}
			\end{figure}
		\item[(b)] For the Causal Bayesian Networks in Figures \ref{fig:cbn1},
			\ref{fig:cbn2} and \ref{fig:cbn3} respectively we have:
			\begin{align}
				p(X,Y) &= p(X)p(Y) \\
				p(X,Y) &= p(Y|X)p(X) \\
				p(X,Y) &= p(X|Y)p(Y)
			\end{align}
		\item[(c)] For the Causal Bayesian Networks in Figures \ref{fig:cbn1}
			and \ref{fig:cbn3} respectively we have:
			\begin{align}
				p(Y|X) &= p(X)p(Y) \\
				p(Y|X) &= \frac{p(X|Y)p(Y)}{p(X)} = \frac{p(X|Y)p(Y)}{\sum_Y p(X|Y)p(Y)}
			\end{align}
			while $p(Y|X)$ is already a term of the factorization for the graph in
			Figure \ref{fig:cbn2}.
		\item[(d)] For the Causal Bayesian Networks in Figures \ref{fig:cbn1},
			\ref{fig:cbn2} and \ref{fig:cbn3} respectively we have:
			\begin{align}
				p(Y|do(X)) &= p(Y) \\
				p(Y|do(X)) &= \frac{p(X,Y)}{p(X)} = p(Y|X) \\
				p(Y|do(X)) &= p(Y)
			\end{align}
		\item[(e)] $p(Y|X)$ is the probability of having lung cancer given the
			\textit{observation} of smoking. $p(Y|do(X))$ instead represents the
			probability of having lung cancer given that we \textit{force} that
			person to (not) smoke. Usual conditioning captures the correlation
			between two random variables whereas the do-operator represents an
			active intervention on the model and is capable of representing
			causation relationships.
	\end{itemize}
\end{sol}

\paragraph{Problem 3. Simpson's paradox}
\begin{sol}
	\begin{itemize}
		\item[1a.] The recovery rate for \textit{treatment} is 50\%, while for
			\textit{untreated} is 40\%.
		\item[1b.] I would advice to take the drug because the recovery rate is
			higher for the \textit{treatment} group.
		\item[2a.]
			\begin{tabular}[H]{ l | l | l }
				\textbf{Recovery rates} & Drug & No drug \\
				\hline
				Male & 60\% & 70\% \\
				Female & 20\% & 30\%
			\end{tabular}
		\item[2b.] I would not advice to take the drug nor to male nor
			female patients because the recovery rate, given the patient's gender,
			does not support it.
		\item[3.] With hindsight I would not advice a patient with unknown gender to
			take the drug because for both genders the recovery rate does not support
			it. This is in contradiction with the conclusion given in (1b).
		\begin{figure}[h]
			\centering
			\begin{tikzpicture}
				\node[obs] (M) at (0,0) {$M$};
				\node[obs] (D) at (-1,-1) {$D$};
				\node[obs] (R) at(1,-1) {$R$};
				\draw[->] (M) -- (D);
				\draw[->] (M) -- (R);
				\draw[->] (D) -- (R);
			\end{tikzpicture}
			\caption{Causal model where $M$ denotes the gender.}
			\label{fig:p3i}
		\end{figure}
		\item[4a.] By applying the back-door criterion on the causal model in Figure
			\ref{fig:p3i} we have:
			\begin{align}
				p(R|do(D)) &= \sum_M p(R|D,M)p(M)
			\end{align}
			because $M$ is admissible for adjustment to find the causal effect of $D$ on $R$.
		\item[4b.] Using normal probability rules we have:
			\begin{align}
				p(R|D) &= \sum_M p(R,M|D) = \sum_M p(R|D,M)p(D|M)
			\end{align}
			which is generally not equal to $\sum_Mp(R|D,M)p(M)$ and so
			$p(R|do(D)) \neq p(R|D)$ in this case.
		\item[4c.] We have expressed $p(R|do(D))$ in terms of observables and we
			have the data to carry on the calculation:
			\begin{align}
				p(R|do(D)) &= p(R|D,M=m)p(M=m) + p(R|D,M=f)p(M=f) = 40\% \\
				p(R|do(\neg D)) &= p(R|\neg D,M=m)p(M=m) + p(R|\neg D,M=f)p(M=f) = 50\%
			\end{align}
			and so in this case I would not advice on taking the drug. The same
			conclusion could have been reached without computation by noting that
			$p(R|do(D)) \neq p(R|D)$.
		\begin{figure}[h]
			\centering
			\begin{tikzpicture}
				\node[obs] (M) at (0,0) {$M$};
				\node[obs] (D) at (-1,-1) {$D$};
				\node[obs] (R) at(1,-1) {$R$};
				\draw[->] (D) -- (M);
				\draw[->] (M) -- (R);
				\draw[->] (D) -- (R);
			\end{tikzpicture}
			\caption{Causal model where $M$ denotes for example ``blood pressure''.}
			\label{fig:p3ii}
		\end{figure}
		\item[5a.] By applying the back-door criterion to the causal model in Figure
			\ref{fig:p3ii} we have:
			\begin{align}
				p(R|do(D)) &= p(R|D)
			\end{align}
			because $\emptyset$ is admissible for adjustment to find the causal effect of $D$ on $R$.
		\item[5b.] Yes, $p(R|do(D)) = p(R|D)$.
		\item[5c.] From the data we have that:
			\begin{align}
				p(R|do(D)) &= p(R|D) = 50\% \\
				p(R|do(\neg D)) &= p(R|\neg D) = 40\%
			\end{align}
			so my advice in this case would be to take the drug. The same conclusion
			could have been reached without computation by noting that $p(R|do(D)) = p(R|D)$.
		\begin{figure}[h]
			\centering
			\begin{tikzpicture}
				\node[obs] (M) at (0,0) {$M$};
				\node[obs] (D) at (-1,-1) {$D$};
				\node[obs] (R) at(1,-1) {$R$};
				\node[nObs] (L1) at (-1,1) {$L_1$};
				\node[nObs] (L2) at (1,1) {$L_2$};
				\draw[->] (D) -- (R);
				\draw[->] (L1) -- (D);
				\draw[->] (L2) -- (R);
				\draw[->] (L1) -- (M);
				\draw[->] (L2) -- (M);
			\end{tikzpicture}
			\caption{Causal model where $M$, $L_1$ and $L_2$ have meanings given in (6a).}
			\label{fig:p3iii}
		\end{figure}
		\item[6a.] We can imagine a causal model where $M$ denotes a certain kind of
			a specific disease (a certain kind of influenza for example), which is
			observed and so we can assume that it is diagnosable. The latent variables
			$L_1$ and $L_2$ might represent whether the patient has a congenital
			disorder that contributes as a cause to the observed disease and the effect
			of the drug ($L_1$) and if among his/her ancestors the same disease is
			present ($L_2$, so this affects the chances of recovery and the kind of
			disease the patient may have). We can suppose that the congenital disorder is not
			(easily) diagnosable and we do not have data for the patient's ancestors.
			The causal model is given in Figure \ref{fig:p3iii}.
		\item[6b.] We can do this in one step by applying the second rule of
			do-calculus, because $(R \ci D)_{\mathcal{G}_{\underline{D}}}$:
			\begin{align}
				p(R|do(D)) &= p(R|D)
			\end{align}
		\item[6c.] Yes, $p(R|do(D)) = p(R|D)$ in this case.
		\item[6d.] My advice in this case would be to take the drug for the same
			reasons given in (5c).
	\end{itemize}
\end{sol}

\newpage

\paragraph{Problem 5. Truncated factorization}
\begin{sol}
	\begin{enumerate}
		\item Assuming that both $Y$ and nodes in $\vt{X}_{pa(X)}$ have no parents
			(this is just an assumption made to have a more uncluttered notation, but
			the result and derivation hold also without it) we have:
			\begin{align}
				p(Y|do(X),\vt{X}_{pa(X)})
				&=\frac{p(Y,\vt{X}_{pa(X)}|do(X))}{p(\vt{X}_{pa(X)}|do(X))} \\
				&=\frac{p(Y)\prod_{X_i \in \vt{X}_{pa(X)}}p(X_i)}{\prod_{X_i \in \vt{X}_{pa(X)}}p(X_i)} \\
				&=p(Y) \\
				&=\frac{p(Y)p(X|\vt{X}_{pa(X)})p(\vt{X}_{pa(X)})}{p(X|\vt{X}_{pa(X)})p(\vt{X}_{pa(X)})} \\
				&=\frac{p(Y,X,\vt{X}_{pa(X)})}{p(X,\vt{X}_{pa(X)})} \\
				&=p(Y|X,\vt{X}_{pa(X)})
			\end{align}
		\item Using the previous result and the truncated factorization for
			$p(\vt{X}_{pa(X)}|do(X))$:
			\begin{align}
				p(Y|do(X))
				&=\int p(Y|do(X),\vt{X}_{pa(X)})p(\vt{X}_{pa(X)}|do(X))\mathrm{d}\vt{X}_{pa(X)} \\
				&=\int p(Y|X,\vt{X}_{pa(X)})\prod_{X_i \in \vt{X}_{pa(X)}}p(X_i)\mathrm{d}\vt{X}_{pa(X)} \\
				&=\int p(Y|X,\vt{X}_{pa(X)})p(\vt{X}_{pa(X)})\mathrm{d}\vt{X}_{pa(X)}
			\end{align}
	\end{enumerate}
\end{sol}


\end{document}

