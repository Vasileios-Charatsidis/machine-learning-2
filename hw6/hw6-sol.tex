\documentclass{amsmlaj}
\begin{document}
\lecturesol{Homework 6}{Ke Tran}{m.k.tran@uva.nl}{19 May, 2016}
{Andrea Jemmett}{11162929}{andrea.jemmett@student.uva.nl}{N/A}


\begin{problem}
\end{problem}

\begin{problem}
\begin{sol}
	\begin{itemize}
		\item[(a)] \hfill \vspace{-1cm}
			\begin{figure}[h]
				\centering
				\begin{subfigure}[b]{.3\textwidth}
					\centering
					\begin{tikzpicture}
						\node[nObs] (x) at (0,0) {$X$};
						\node[nObs] (y) at (1.5,0) {$Y$};
					\end{tikzpicture}
					\caption{No cause-effect relation}
					\label{fig:cbn1}
				\end{subfigure}
				\begin{subfigure}[b]{.3\textwidth}
					\centering
					\begin{tikzpicture}
						\node[nObs] (x) at (0,0) {$X$};
						\node[nObs] (y) at (1.5,0) {$Y$};
						\draw[->] (x) -- (y);
					\end{tikzpicture}
					\caption{$X$ causes $Y$}
					\label{fig:cbn2}
				\end{subfigure}
				\begin{subfigure}[b]{.3\textwidth}
					\centering
					\begin{tikzpicture}
						\node[nObs] (x) at (0,0) {$X$};
						\node[nObs] (y) at (1.5,0) {$Y$};
						\draw[->] (y) -- (x);
					\end{tikzpicture}
					\caption{$Y$ causes $X$}
					\label{fig:cbn3}
				\end{subfigure}
				\caption{Two nodes Causal Bayesian Networks}
				\label{fig:cbn}
			\end{figure}
		\item[(b)] For the Causal Bayesian Networks in Figures \ref{fig:cbn1},
			\ref{fig:cbn2} and \ref{fig:cbn3} respectively we have:
			\begin{align}
				p(X,Y) &= p(X)p(Y) \\
				p(X,Y) &= p(Y|X)p(X) \\
				p(X,Y) &= p(X|Y)p(Y)
			\end{align}
		\item[(c)] For the Causal Bayesian Networks in Figures \ref{fig:cbn1}
			and \ref{fig:cbn3} respectively we have:
			\begin{align}
				p(Y|X) &= p(X)p(Y) \\
				p(Y|X) &= \frac{p(X|Y)p(Y)}{p(X)} = \frac{p(X|Y)p(Y)}{\sum_Y p(X|Y)p(Y)}
			\end{align}
			while $p(Y|X)$ is already a term of the factorization for the graph in
			Figure \ref{fig:cbn2}.
		\item[(d)] For the Causal Bayesian Networks in Figures \ref{fig:cbn1},
			\ref{fig:cbn2} and \ref{fig:cbn3} respectively we have:
			\begin{align}
				p(Y|do(X)) &= p(Y) \\
				p(Y|do(X)) &= \frac{p(X,Y)}{p(X)} = p(Y|X) \\
				p(Y|do(X)) &= p(Y)
			\end{align}
		\item[(e)] % TODO
	\end{itemize}
\end{sol}
\end{problem}

\begin{problem}{\textbf{Simpson's paradox}}
\begin{sol}
	\begin{itemize}
		\item[1a.] The recovery rate for \textit{treatment} is 50\%, while for
			\textit{untreated} is 40\%.
		\item[1b.] I would advise to take the drug because the recovery rate is
			higher for the \textit{treatment} group.
		\item[2a.]
			\begin{tabular}[H]{ l | l | l }
				\textbf{Recovery rates} & Drug & No drug \\
				\hline
				Male & 60\% & 70\% \\
				Female & 20\% & 30\%
			\end{tabular}
		\item[2b.] I would not advice to take the drug nor to male patients nor to
			female patients because the recovery rate, given the patient's gender supports it.
		\item[3.] With hindsight I would not advice a patient with unknown gender to
			take the drug because for both genders the recovery rate does not support
			it. This is in contradiction with the conclusion given in (1b).
		\begin{figure}[H]
			\centering
			\begin{tikzpicture}
				\node[obs] (M) at (0,0) {$M$};
				\node[obs] (D) at (-1,-1) {$D$};
				\node[obs] (R) at(1,-1) {$R$};
				\draw[->] (M) -- (D);
				\draw[->] (M) -- (R);
				\draw[->] (D) -- (R);
			\end{tikzpicture}
			\caption{Causal model where $M$ denotes the gender.}
			\label{fig:p3i}
		\end{figure}
		\item[4a.] By applying the back-door criterion on the causal model in Figure
			\ref{fig:p3i} we have:
			\begin{align}
				p(R|do(D)) &= \sum_M p(R|D,M)p(M)
			\end{align}
		\item[4b.] Using normal probability rules we have:
			\begin{align}
				p(R|D) &= \sum_M p(R,M|D) = \sum_M p(R|D,M)p(M)
			\end{align}
			and so $p(R|do(D)) = p(R|D)$ in this case.
	\end{itemize}
\end{sol}
\end{problem}


\end{document}

